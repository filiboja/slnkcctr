% `\input` in a section

% Author: Filip Bartek

\subsection{Beat detector}

Beat detector collects a history (sliding window) of detected
slinky end positions and estimates occurences of steps,
i.e.~moments when the slinky end hits the juggler's hand.

These moments are very easy to determine for a human
observing the juggler
and quite easy to determine for a human reading through a history of
detected slinky end positions,
but implementing a similar functionality in a computer
proved to be quite difficult.

The main factors that contribute to the difficulty
of beat time estimation are:

\begin{itemize}

\item low time resolution of sampled positions
(time resolution is limited by camera frame rate),

\item low precision of sampled positions
(caused by low precision of slinky end detection process,
described in section \ref{sec:slinkyenddetector}) and

\item noise in sampled positions
(caused namely by parts of slinky obscuring the ends while juggling).

\end{itemize}

In the end we resorted to implementing a simple method
optimized for a 15 FPS camera
and a slinky with step period between
$\frac{6}{15}$ and $\frac{12}{15}$ s.
These values were chosen to work with the slinkys and camera
we used for testing
and we believe that
most common slinkys and cameras satisfy these assumptions.

As mentioned above,
the beat detection process proved to be more difficult to implement
than we originally expected
and the current implementation presents a notable room for improvement.
