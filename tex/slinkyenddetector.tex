% `input` in a section

% Author: Filip Bartek

\subsection{Slinky end detector}
\label{sec:slinkyenddetector}
% http://opencv-srf.blogspot.com/2010/09/object-detection-using-color-seperation.html

Slinky end detection is performed using color filtering.
OpenCV library is used in this step extensively.

The detector takes an image and outputs an estimated position
of an end of slinky in the image.
If slinky end is not detected, a non-position is output.

The detector consists of four filters:

\begin{itemize}
\item \texttt{CropFilter} (crops the image)
\item \texttt{HsvFilter} (filters out some colors)
\item \texttt{OpenFilter} (morphological opening)
\item \texttt{CloseFilter} (morphological closing)
\end{itemize}

The filters select a part of the image, i.e.~a set of pixels.
Slinky end position is determined as the centroid of this set.

Each of the filters is parametrized by one or more numeric values.
The parameters are exposed in \gls{gui} and configuration files.
Refer to the \slnkcctr{} documentation\cite{bartek:slnkcctrreadme}
for details.
