% `\input` in any level

% Author: Filip Bartek

\begin{minipage}{\linewidth}
Project proposal\cite{bartek:proposal} describes the basic desired
behavior of \slinkyconductor{}:

\begin{itemize}
\item when slinky is idle, so is \slinkyconductor{} and
\item when slinky is ``walking'' down the hands of a juggler, \slinkyconductor{} plays back music synchronized to the steps of the slinky.
\end{itemize}
\end{minipage}

We decided to pursue this behavior by implementing
a system that consists of
a computer that runs specialized software (called \slnkcctr{}) and has
a video camera and a sound output device connected to it.
The system (in real-time) tracks the position of slinky
in image from camera,
recognizes the phase in the slinky ``walking'' cycle and
plays back music synchronized to this recognized phase.

% TODO: Make a picture of components here.
% * computer
%     * slnkcctr software
% * camera
% * speakers
% [FB] Suggestion: Use ntikz package.

% TODO: Make a picture of software components here with information
% flow (maybe).
% * slinky position tracker
% * beat recognizer
% * sound player

We decided to use this hardware setup because we feel it's
familiar and affordable for many computer users.
This approach also complements the low-price policy
the original Slinky toy followed throughout most of its history
on behalf of Betty James,
the toy's inventor's wife.\cite[section History]{wiki:slinky}
Indeed, slinky remains a notably affordable toy.\footnote{
As of now,
a package of 3 original Slinkys can be obtained
for \$12.99\cite{poof:slinky}
and the shop \emph{TojeTo} in Ljubljana carries Magic Spring,
a good plastic Slinky duplicate,
for \EUR{2.90}\cite{tojeto:slinky}.
}

For similar reasons, we decided to keep the project
open source\footnote{
The source code of \slnkcctr{} is publicly available
from the \emph{GitHub}
repository \texttt{filiboja/slnkcctr}\cite{filiboja:slnkcctr}.
}
and free to use.

To facilitate visual tracking of the slinky,
we decided to use a brightly colored plastic slinky
instead of the traditional metal variant.
For practical reasons, we further require the slinky to have
different colors at its ends,
and for the colors to be easily separable from background.
We believe that these properties are not uncommon because
most slinkys available around are rainbow-colored
% TODO: Rewrite in a less informal manner.
(see an example in figure \ref{fig:slinky-magic}).
